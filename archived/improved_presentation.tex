
\documentclass[aspectratio=169]{beamer}

\usetheme{Madrid}
\usecolortheme{beaver} % A slightly more professional color scheme (Red/Grey)
\setbeamertemplate{navigation symbols}{}
\setbeamertemplate{itemize items}[circle]

\usepackage{amsmath,amssymb}
\usepackage{graphicx}
\usepackage{booktabs}
\usepackage{xcolor}

% Custom Colors regarding MACE/CHGNet
\definecolor{maceblue}{RGB}{0, 80, 150}
\definecolor{alertred}{RGB}{200, 0, 0}

\title[LLZO/Li ML Reliability]{Reliable Active Learning for LLZO/Li Interfaces}
\subtitle{Bridging the Gap: From Universal Surrogates to Specialized Fine-Tuning}
\author{Mehul}
\institute{IISc}
\date{\today}

\begin{document}

%------------------------------------------------
\begin{frame}
\titlepage
\end{frame}

%------------------------------------------------
\begin{frame}{1. The Challenge: Combinatorial Complexity}
\begin{columns}
\column{0.6\textwidth}
\begin{itemize}
    \item \textbf{Goal:} Study the Li-metal / Solid Electrolyte (LLZO) interface with MLIPs.
    \item \textbf{The Problem:} The phase space is combinatorial:
    \begin{itemize}
        \item \textbf{Strain:} Lattice mismatch drives instabilities.
        \item \textbf{Chemistry:} 6+ distinct surface terminations (La, Zr, Li-rich...).
        \item \textbf{Geometry:} Multiple facet orientations (100, 110, 111).
    \end{itemize}
    \item \textbf{Bottleneck:} DFT is too expensive ($\sim$1000s of CPU-hours/cal) to explore this fully.
\end{itemize}
\column{0.38\textwidth}
\begin{alertblock}{Key Constraint}
We cannot run DFT on everything. We need a reliable filter to run DFT on.
\end{alertblock}
\end{columns}
\end{frame}

%------------------------------------------------
\begin{frame}{2. The Solution: A Surrogate-First Strategy}
\textbf{Phase 1: High-Throughput Generation (The "Wide Net")}
\begin{itemize}
    \item Use \textbf{CHGNet} (Universal Model) as a zero-shot surrogate.
    \item \textbf{Action:} Generated \& Relaxed \textbf{8,654} interface structures.
    \item \textbf{Coverage:} 
    \begin{itemize}
        \item Strains: $0\%, \pm1\%, \pm2\%, \pm3\%$
        \item Temperatures: MD Snapshots at 300K, 450K.
    \end{itemize}
\end{itemize}

\vspace{1em}
\textbf{Phase 2: Specialized Reliability (The "Filter")}
\begin{itemize}
    \item Fine-tuned **MACE** models on subsets of this data.
    \item \textbf{Why?} Universal models are "Jacks of all trades." We need a Master of \emph{this} interface.
\end{itemize}
\end{frame}

%------------------------------------------------
\begin{frame}{3. Dataset Architecture}
We curated three distinct datasets to test generalization, not just accuracy.

\begin{table}[]
\centering
\begin{tabular}{l l l}
\toprule
\textbf{Dataset} & \textbf{Role} & \textbf{Count} \\
\midrule
\textbf{T1} & \textbf{Training Set A} (Primary Interface) & 6,337 \\
\textbf{T2} & \textbf{Validation Set} (Held-out, Safe) & 705 \\
\textbf{T3} & \textbf{Training Set B} (Alternate Interface/Strain) & 1,612 \\
\bottomrule
\end{tabular}
\end{table}

\vspace{0.5em}
\textbf{Model Strategy:}
\begin{itemize}
    \item \textbf{MACE\_T1:} Trained on T1 $\to$ Tested on T3 (OOD test).
    \item \textbf{MACE\_T2:} Trained on T3 $\to$ Tested on T1 (OOD test).
    \item \textbf{Goal:} If models disagree, the structure is objectively unreliable.
\end{itemize}
\end{frame}

%------------------------------------------------
\begin{frame}{4. The Safety Mechanism: Dual OOD Detection}
We assume the ML prediction is \textbf{wrong} unless proven otherwise by two independent physics checks:

\begin{columns}
\column{0.48\textwidth}
\begin{block}{1. Epistemic Uncertainty ($u_E$)}
\emph{"Does the model confuse itself?"}
\begin{itemize}
    \item Calculated via \textbf{Deep Ensemble Variance} (4 seeds).
    \item Captures \textbf{Label Noise} and complex physics.
\end{itemize}
\end{block}

\column{0.48\textwidth}
\begin{block}{2. Latent Distance ($d_{latent}$)}
\emph{"Is the geometry weird?"}
\begin{itemize}
    \item Euclidean distance in PCA-reduced MACE latent space.
        \item Captures \textbf{Geometric Novelty} (e.g., bond breaking).
\end{itemize}
\end{block}
\end{columns}

\vspace{1em}
\textbf{Hypothesis:} An OOD structure will trigger \emph{at least one} of these flags.
\end{frame}

%------------------------------------------------
\begin{frame}{Mathematical Definition: Latent-Space Distance ($d_{latent}$)}

\textbf{Latent Representation}
\begin{itemize}
    \item Each structure $X$ with $N$ atoms is mapped by MACE to atomic embeddings:
    \[
    \mathbf{z}_i \in \mathbb{R}^{128}, \quad i = 1,\dots,N
    \]
    \item Structure-level latent vector obtained by mean pooling:
    \[
    \mathbf{z}(X) = \frac{1}{N} \sum_{i=1}^{N} \mathbf{z}_i
    \]
\end{itemize}

\vspace{0.5em}
\textbf{Dimensionality Reduction}
\begin{itemize}
    \item Stack training latents $\{\mathbf{z}(X_j)\}$ into matrix $Z \in \mathbb{R}^{N_s \times 128}$
    \item Apply PCA:
    \[
    \mathbf{y}(X) = W^\top (\mathbf{z}(X) - \boldsymbol{\mu}), \quad W \in \mathbb{R}^{128 \times k}
    \]
    \item Choose $k=2$ (captures $>99\%$ variance)
\end{itemize}

\textbf{Latent Distance}
\[
d_{\text{latent}}(X) = \left\| \mathbf{y}(X) - \bar{\mathbf{y}}_{\text{train}} \right\|_2
\]

\end{frame}
%------------------------------------------------
\begin{frame}{Mathematical Definition: Energy Uncertainty ($u_E$)}

\textbf{Ensemble Predictions}
\begin{itemize}
    \item Consider an ensemble of $M$ independently trained MACE models:
    \[
    \{ f^{(1)}, f^{(2)}, \dots, f^{(M)} \}
    \]
    \item Each model predicts a total energy:
    \[
    E^{(m)}(X) = f^{(m)}(X)
    \]
\end{itemize}

\textbf{Ensemble Statistics}
\[
\bar{E}(X) = \frac{1}{M} \sum_{m=1}^{M} E^{(m)}(X)
\]

\[
\sigma_E^2(X) = \frac{1}{M} \sum_{m=1}^{M} \left( E^{(m)}(X) - \bar{E}(X) \right)^2
\]

\textbf{Size-Normalized Uncertainty}
\[
u_E(X) = \frac{\sigma_E(X)}{N}
\]

\textbf{Interpretation:}
\begin{itemize}
    \item Measures epistemic uncertainty (model disagreement)
    \item High $u_E$ $\Rightarrow$ extrapolation or unresolved physics
\end{itemize}

\end{frame}
%------------------------------------------------
\begin{frame}{Why Two Metrics? (Mathematically)}
\begin{itemize}
    \item $d_{latent}(X)$ measures \textbf{geometric novelty} in representation space.
    \item $u_E(X)$ measures \textbf{functional instability} of the learned energy map.
    \item Agreement between the two implies:
\end{itemize}

\[
\text{Novel Geometry} \;\Rightarrow\; \text{Unstable Energy Prediction}
\]

\vspace{0.5em}
\textbf{This links structure $\rightarrow$ representation $\rightarrow$ energy.}
\end{frame}

%------------------------------------------------
\begin{frame}{Definition of OOD in This Work}
\textbf{Out-of-Distribution (OOD)} in this project means:

\vspace{0.5em}
\begin{itemize}
\item A structure whose latent representation lies far from the high-density core 
of the learned manifold of the finetuned models.
\end{itemize}

\vspace{0.5em}
\textbf{Operational Definition:}
\begin{itemize}
\item Thresholds are defined via training-set percentiles:
\[
\tau_{99} = \mathrm{Quantile}_{0.99}\left( \{ u_E(X_j) \}_{X_j \in D_{\text{train}}} \right)
\]
(similarly for $d_{latent}$)

    \item A structure is flagged OOD if it exceeds the \textbf{99\% percentile}
    \item of the training distribution in either:
    \begin{itemize}
        \item Energy uncertainty ($u_E$), or
        \item Latent-space distance ($d_{latent}$).
    \end{itemize}
\end{itemize}

\vspace{0.5em}
\textbf{Interpretation:}
\begin{itemize}
    \item OOD $\neq$ wrong prediction
    \item OOD $\Rightarrow$ extrapolation risk $\Rightarrow$ candidate for DFT
\end{itemize}
\end{frame}

%------------------------------------------------
\begin{frame}{5. Screening Results (The "Funnel")}
\textbf{Why we chose the 99\% Threshold:}
\begin{itemize}
    \item \textbf{90\% Threshold:} $>1200$ flags (Captures valid thermal noise $\to$ Too Noisy).
    \item \textbf{95\% Threshold:} $\sim 400$ flags (Many redundant trajectory snapshots).
    \item \textbf{99\% Threshold (Strict):} \textbf{120 flags}. Isolates true physical breakdown events.
\end{itemize}

\vspace{0.5em}
\textbf{Final OOD Pools (at 99\%):}
\begin{itemize}
    
    \item \textbf{Combined Union (Union of $u_E \cup d_{latent}$):}
    \begin{itemize}
        \item Total Unique Count: \textbf{120 structures}.
        \item Represents the full "Screening Pool" for active learning.
    \end{itemize}
    
    \item \textbf{Intersection ($u_E \cap d_{latent}$):}
    \begin{itemize}
        \item Count: \textbf{17 structures}. (Too conservative).
    \end{itemize}
\end{itemize}


\centering
\textbf{Which set do we send to DFT?}
\end{frame}

%------------------------------------------------
\begin{frame}{6. Verification: Dimensionality \& Disagreement}
\begin{columns}
\column{0.4\textwidth}
\textbf{1. The Physics is Low-Dimensional}
\begin{center}
    \includegraphics[width=0.95\textwidth]{Thesis_Results/Thesis_PCA_Scree_Plot.png}
\end{center}
\small
\emph{Result:} $>99\%$ of variance is captured by just 2 dimensions. The OOD landscape is structurally simple.

\column{0.4\textwidth}
\textbf{2. OOD $\implies$ Model Collapse}
\begin{center}
    \includegraphics[width=0.95\textwidth]{Thesis_Results/Thesis_Model_Disagreement.png}
\end{center}
\small
\textbf{Metric:} \emph{Uncertainty Divergence} $\Delta u_E = |u_{E}^{T1} - u_{E}^{T3}|$.
\emph{Result:} OOD structures cause models to disagree on their confidence levels (Consensus Break).
\emph{Interpretation:} Some geometrically novel structures remain physically stable,
leading to high $d_{latent}$ but low $u_E$.

\end{columns}
\end{frame}

%------------------------------------------------
\begin{frame}{7. The "Physics-Aware" Selection Strategy}
\begin{alertblock}{Problem}
The Intersection (17) is too small (risk of missing edge cases).\\
The Union (120) contains redundant snapshots of similar trajectories.
\end{alertblock}

\textbf{Solution: Metadata Clustering}
We group the 120 Union structures by their defining \textbf{Physical Boundary Conditions}:
\begin{itemize}
    \item \textbf{1. Mechanical:} Strain State ($0\%, \pm1\%, \pm2\%, \pm3\%$)
    \item \textbf{2. Chemical:} Surface Termination (La, Zr, Li-rich, etc.)
    \item \textbf{3. Geometric:} Facet Orientation (100, 110, 111)
\end{itemize}

\vspace{0.5em}
\textbf{The Logic:} Structures sharing these 3 tags represent the \emph{same physical interface}. Variations are just thermal noise.
\begin{itemize}
    \item \textbf{Result:} 120 Structures $\to$ \textbf{19 Unique Physical Conditions}.
\end{itemize}

\end{frame}

%------------------------------------------------
\begin{frame}{8. Selected DFT Candidates (The 19 Conditions)}
The 120 OOD structures collapsed into \textbf{5 Primary Interface Classes}, covering 19 specific strain states.

\begin{table}[]
\centering
\small
\begin{tabular}{l l l l}
\toprule
\textbf{Class} & \textbf{Termination} & \textbf{Facet} & \textbf{Strains Covered (Unique Conditions)} \\
\midrule
\textbf{I} & La & 110 & $0\%, \pm1\%, \pm2\%, \pm3\%, -1.5\%$ (8 States) \\
\textbf{II} & La & 111 & $0\%, +1\%, +1.5\%, +2\%, +3\%$ (5 States) \\
\textbf{III} & Zr & 111 & $+2\%, +3\%$ (High Tensile Only) \\
\textbf{IV} & Li-Rich & 100 & $+2\%, +3\%$ (High Tensile Only) \\
\textbf{V} & La & 100 & $+2\%, +3\%$ (High Tensile Only) \\
\bottomrule
\end{tabular}
\end{table}

\vspace{0.5em}
\textbf{Physical Insight:}
\begin{itemize}
    \item \textbf{La-110 (Class I)} is the most unstable interface, failing across almost the entire strain range.
    \item \textbf{Zr \& Li-Rich (Classes III-V)} are generally stable, failing only at extreme tensile strains ($>2\%$).
\end{itemize}
\end{frame}
%------------------------------------------------
\begin{frame}{9. Appendix: Full List of 19 Candidates}
\tiny
\begin{columns}[t]
\column{0.48\textwidth}
\textbf{Group I: La-Terminated (110) -- Highly Unstable}
\begin{table}[]
\begin{tabular}{l l l}
\toprule
\textbf{ID} & \textbf{Condition} & \textbf{Status} \\
\midrule
1 & La (110) Strain 0.0\% & OOD \\
2 & La (110) Strain +1.0\% & OOD \\
3 & La (110) Strain +1.5\% & OOD \\
4 & La (110) Strain +2.0\% & OOD \\
5 & La (110) Strain +3.0\% & OOD \\
6 & La (110) Strain -1.0\% & OOD \\
7 & La (110) Strain -2.0\% & OOD \\
8 & La (110) Strain -3.0\% & OOD \\
\bottomrule
\end{tabular}
\end{table}

\column{0.48\textwidth}
\textbf{Group II: La-Terminated (111)}
\begin{table}[]
\begin{tabular}{l l l}
\toprule
\textbf{ID} & \textbf{Condition} & \textbf{Status} \\
\midrule
9 & La (111) Strain 0.0\% & OOD \\
10 & La (111) Strain +1.0\% & OOD \\
11 & La (111) Strain +1.5\% & OOD \\
12 & La (111) Strain +2.0\% & OOD \\
13 & La (111) Strain +3.0\% & OOD \\
\bottomrule
\end{tabular}
\end{table}

\textbf{Group III-V: Stable Interfaces (Only High Tension)}
\begin{table}[]
\begin{tabular}{l l l}
\toprule
\textbf{ID} & \textbf{Condition} & \textbf{Status} \\
\midrule
14-15 & Zr (111) Strain +2\%, +3\% & OOD \\
16-17 & Li-Rich (100) Strain +2\%, +3\% & OOD \\
18-19 & La (100) Strain +2\%, +3\% & OOD \\
\bottomrule
\end{tabular}
\end{table}
\end{columns}
\end{frame}

%------------------------------------------------
\begin{frame}{10. Final Action Plan: Targeted Active Learning}
Step 1: \textbf{Compute DFT} for the \textbf{19 Representative Candidates}.
\begin{itemize}
    \item Cost: Negligible (~20 DFTs vs 8600).
    \item Coverage: Spans every failure mode discovered (Strain + Geometry).
\end{itemize}

Step 2: \textbf{Fine-Tune Universal Model}
\begin{itemize}
    \item Add these 19 "Hard Negatives" to the training set.
    \item Retrain MACE Universal.
\end{itemize}

Step 3: \textbf{Verify Contraction}
\begin{itemize}
    \item Re-run the OOD scan. The "OOD Region" should shrink, confirming the model has learned the new physics.
\end{itemize}
\end{frame}

%------------------------------------------------
\begin{frame}{Summary}
\begin{enumerate}
    \item **Scale:** Generated 8600+ interface structures via surrogate.
    \item **Reliability:** Built a dual-check system (Uncertainty + Geometry).
    \item **Validation:** Proven that OOD flags correlate with Model Disagreement.
    \item **Efficiency:** Reduced 120 potential errors to \textbf{19 DFT calculations} without losing physical coverage.
\end{enumerate}

\centering
\vspace{1em}
\textbf{This creates a closed-loop, reliable pipeline for solid-state interface discovery.}
\end{frame}

\end{document}
